% Package
\documentclass[11pt]{article}

\usepackage{amsmath}
\usepackage{cite}
\usepackage{graphicx}
\usepackage[utf8]{inputenc}
\usepackage[T1]{fontenc}
\usepackage{lmodern}
\usepackage[ngerman]{babel}
\usepackage{hyphenat}

\title{ADP Aufgabe 2 Entwurf, Abgabe 1}
\author{Team 1\\Hugo Protsch, Justin Hoffmann}

% Document
\begin{document}

    \maketitle

    \tableofcontents


    \section*{Formales}\label{sec:Formales}

    %! suppress = MissingLabel

    \subsection*{Aufgabenaufteilung}
    Der Entwurf für Insertion Sort wurde zusammen entwickelt.\\
    Der Entwurf für Quick Sort wurde von Hugo entwickelt.\\
    Der Entwurf für Heap Sort wurde von Justin entwickelt.
    %! suppress = MissingLabel

    \subsection*{Quellenangaben}
    Es wurden lediglich Vorlesungsmaterialien verwendet.
    %! suppress = MissingLabel

    \subsection*{Bearbeitungszeitraum}
    Der gesamte Arbeitsaufwand für den Entwurf belief sich auf ca. 6 Stunden.
    %2 Organisatorisches +
    %4 Insertion Sort +
    %TODO Quicksort +
    %TODO Heapsort

    %! suppress = MissingLabel

    \subsection*{Aktueller Stand}
    %! suppress = MissingLabel

    \subsection*{Änderungen des Entwurfes}
    -- nicht zutreffend --

    \newpage


    \section{Insertion Sort}\label{sec:insertion-sort}
    Siehe Abbildung~\ref{fig:insertionS}.
    Bei Insertion Sort wird eine Liste durch das Einfügen von Elementen aus
    einem unsortierten Bereich (zunächst die komplette List) in einen sortieren
    Bereich (zunächst leer, <N1>) sortiert.

    Bei dem Einfügen eines Elements E in den sortierten Bereich muss dabei
    jeweils der Bereich bis zu dem Element durchlaufen werden, hinter das das
    Element E eingefügt werden muss (Siehe "Insert into sorted list" Subgraph).

    Somit wird der sortierte Bereich mit jeder Iteration um 1 erhöht <E1>.
    Sobald der sortierte Bereich alle Elemente enthält, wird die Liste
    zurückgegeben.
    Bei der Implementation des Algorithmus auf einfach verkettete Listen
    kann der sortierte und unsortierte Bereich getrennt werden.

    \begin{figure}[hbt]
        \caption{Insertion Sort}
        \centering
        \includegraphics[width = 8cm]{insertionS}\label{fig:insertionS}
    \end{figure}

\end{document}
